\documentclass[12pt]{article}

\usepackage{sbc-template}
\usepackage[brazil,american]{babel}
\usepackage[utf8]{inputenc}

\usepackage{graphicx}
\usepackage{url}
\usepackage{float}
\usepackage{listings}
\usepackage{color}
\usepackage{todonotes}
\usepackage{algorithmic}
\usepackage{algorithm}
\usepackage{hyperref}
\usepackage{indentfirst}

\sloppy

\title{Laboratório 1\\- Assembly MIPS –}

\author{Dayanne Fernandes da Cunha, 13/0107191}
\author{Lucas Mafra Chagas, 12/0126443}
\author{Marcelo Giordano Martins Costa de Oliveira, 12/0037301}



\address{Dep. Ciência da Computação -- Universidade de Brasília (UnB)\\
  CiC 116394 - OAC - Turma A
  \email{}
}

\begin{document}
\maketitle

\selectlanguage{american}
 \begin{abstract}
	This report corresponds to the Experiment 1 about Assembly MIPS.
 \end{abstract}
\selectlanguage{brazil}
 \begin{resumo}
	Este relatório corresponde ao Experimento 1 sobre Assembly MIPS.
 \end{resumo}

\section{Introdução}
\label{sec:Introducao}

\subsection{Objetivos}
\label{sec:Objetivos}

\begin{itemize}
\item Familiarizar o aluno com o Simulador/Montador MARS;
\item Desenvolver a capacidade de codificação de algoritmos em linguagem Assembly MIPS;
\item Desenvolver a capacidade de análise de desempenho de algoritmos em Assembly;
\end{itemize}

\subsection{Ferramentas}
\label{sec:Materiais}

\begin{itemize}
\item MARS v.4.5 Custom 7
\item Cross compiler MIPS GCC
\item Inkscape e GIMP
\end{itemize}

\section{Procedimentos}
\label{sec:Procedimentos}

\subsection{Simulador/Montador MARS}
\label{subsec:mars}
Essa parte do relatório foi realizada com o intuito de familiarizar os alunos ao Simulador/Montador MARS.

No item 1.2 do relatório, foram pedidos os gráficos relacionados aos valores do vetor fornecido e ao tempo de execução da subrotina Sort fornecida pelo professor. Temos então o grafico em Melhor caso~\ref{fig:txnMC} e Pior Caso~\ref{fig:txnPC} da rotina de ordenação Sort.

\begin{figure}[H]
	\centering
	\includegraphics[width=1\textwidth]{txnMC.png}
	\caption{n x texec(Melhor Caso)}
	\label{fig:txnMC}
\end{figure}

\begin{figure}[H]
	\centering
	\includegraphics[width=1\textwidth]{txnPC.png}
	\caption{n x texec(Pior Caso)}
	\label{fig:txnPC}
\end{figure}

Analisando somente os gráficos, percebemos que esse algoritmo tem tempo de execução Linear em seu Melhor caso e em Pior caso, sendo representado por um braço de parábola, tem tempo de execução limitado por O(\(n^2\)). 

\subsection{Compilador GCC}
\label{subsec:comp}

Para compilar código C em código Assembly foi utilizado o cross compiler \cite{MIPS} GCC. Usando o comando \$ mips-elf-gcc -I../include -S testeX.c (X $\in$ [0,8]) foi testado a convenção para geração do código Assembly.

\subsubsection{Diretivas}
\label{subsubsec:diretivas}

Diretivas são apenas comandos ao montador e não fazem parte do conjunto de instruções dos processadores \textit{x86}. Todas as diretivas começam com (.) (\textit{ASCII 0x2E}). Elas permitem a alocação de espaço para a declaração de variáveis " \textit{.byte}, \textit{.word} ", definição de escopo " \textit{.glob1} ", além de várias outras funções de gerenciamento como as listadas a seguir (as informações abaixo foram encontradas usando as fontes \cite{mips1}, \cite{mips2-1}, \cite{mips2-2}, \cite{mips2-3}, \cite{mips3},
\cite{mips4}, \cite{mips5} e \cite{mips6} ):

\begin{itemize}

\item .file \textit{string} : Cria uma tabela de símbolos de entrada onde a \textit{string} é o nome do símbolo e \textit{STT\_FILE} é o tipo deste símbolo, a \textit{string} especifica o nome do arquivo fonte associado ao arquivo objeto.
\item .section \textit{section}, \textit{attributes} : \textit{Section} é montado como seção atual. \textit{Attributes} é incluso se for a primeira vez que \textit{.section} é especificado.
\item .mdebug : Força a saída de depuração para entrar em uma seção .mdebug de estilo ECOFF em vez das seções padrão ELF .stabs.
\item .previous : Troca esta seção pela que foi referenciada recentemente.
\item .nan : Esta diretiva diz qual codificação \textit{MIPS} será usada para ponto flutuante IEEE 754. A primeira, padrão \textit{2008}, diz para o montador utilizar a codificação IEEE 754-2008, enquando a \textit{legacy} utiliza a codificação original do \textit{MIPS}.
\item .gnu\_attribute \textit{tag}, \textit{value} : Grava um atributo objeto \textit{gnu} para este arquivo.
\item .globl \textit{symbol1, symbol2, ..., symbolN}: Torna global cada símbolo da lista. A diretiva torna o símbolo global no escopo mas não declara o símbolo.
\item .data : Muda a seção atual para \textit{.data} (dados estáticos do programa).
\item .type \textit{symbol[, symbol, ..., symbol], type[, visibility]}: Atribui tipo ao símbolo, podendo ser do tipo função, objeto, sem tipo e um objeto \textit{TLS (Thread Local Storage)}.
\item .size \textit{symbol, expr}: Resolve expressão e atribui tamanho em bytes ao \textit{symbol}.
\item .word : Armazena o valor listado como palavras de 32 bits no limite.
\item .rdata : Adiciona dados apenas de leitura.
\item .align \textit{integer} : Ajusta o contador de locação para um valor múltiplo de 2.
\item .ascii "\textit{string}" : Aloca espaço para cadeias de caracteres sem o "\textit{$\backslash$0}".
\item .text : Muda a seção atual para \textit{.text} (instruções).
\item .ent \textit{name[,label]} : Marca o começo da função \textit{name}.
\item .frame : Descreve o quadro da pilha usada para chamar a função principal(\textit{main}).
\item .set \textit{symbol, expression}: Resolve a expressão (\textit{expression}) e atribui o valor ao símbolo (\textit{symbol}).
\item .mask \textit{mask offset} : Configura uma máscara que indica quais registradores de uso geral foram salvos na rotina atual. Esses valores são usados pelo montador para gerar a seção \textit{.reginfo} do arquivo objeto dos processadores \textit{MIPS}.
\item .fmask \textit{mask offset} : Configura uma máscara informando os registradores de ponto flutuante que a rotina atual salvou. Esses valores são usados pelo montador para gerar a seção \textit{.reginfo} do arquivo objeto dos processadores \textit{MIPS}.

\end{itemize}

\subsubsection{Assembly no MARS}
\label{subsubsec:atomars}

Algumas diretivas listadas acima não são reconhecidas pelo \textit{MARS (Mips Assembly and Runtime Simulator)}, como por exemplo \textit{.section, .previous, . nan, etc}, assim como alguns elementos como \textit{@object, @function, etc}. Logo, as seguintes instruções foram retiradas:

\begin{itemize}
\item @object\\
  \begin{itemize}
  \item .type v, @object
  \end{itemize}
 \item @function\\
  \begin{itemize}
  \item .type show, @function
  \item .type swap, @function
  \item .type sort, @function
  \item .type main, @function
  \end{itemize}
\item .section\\
  \begin{itemize}
  \item .section .mdebug.abi32
  \end{itemize}
\item .previous\\
  \begin{itemize}
  \item .previous
  \end{itemize}
\item .nan\\
  \begin{itemize}
  \item .nan legacy
  \end{itemize}
\item .gnu\_attribute\\
  \begin{itemize}
  \item .gnu\_attribute 4, 1
  \end{itemize}
\item .size\\
  \begin{itemize}
  \item .size v, 40
  \item .size show, .-show
  \item .size swap, .-swap
  \item .size sort, .-sort
  \item .size	main, .-main
  \end{itemize}
\item .rdata\\
  \begin{itemize}
  \item .rdata
  \end{itemize}
\item .set\\
  \begin{itemize}
  \item .set nomips16
  \item .set nomicromips
  \item .set noreorder
  \item .set nomacro
  \item .set reorder
  \item .set macro
  \end{itemize}
\item .ent\\
  \begin{itemize}
  \item .ent show
  \item .ent swap
  \item .ent sort
  \item .ent main
  \end{itemize}
\item .frame\\
  \begin{itemize}
  \item .frame \$fp,24,\$31 \# vars= 0, regs= 2/0, args= 16, gp= 0
  \item .frame \$fp,32,\$31 \# vars= 8, regs= 2/0, args= 16, gp= 0
  \item .frame \$fp,16,\$31 \# vars= 8, regs= 1/0, args= 0, gp= 0
  \item .frame \$fp,32,\$31 \# vars= 8, regs= 2/0, args= 16, gp= 0
  \end{itemize}
\item .mask\\
  \begin{itemize}
  \item .mask	0xc0000000,-4
  \end{itemize}
\item .fmask\\
  \begin{itemize}
  \item .fmask	0x00000000,0
  \end{itemize}
\item .end\\
  \begin{itemize}
  \item .end swap
  \item .end sort
  \item .end main
  \item .end show
  \end{itemize}
\item .ident\\
  \begin{itemize}
  \item .ident "GCC: (GNU) 4.8.1"
  \end{itemize}
\end{itemize}

Há também outras instruções que o simulador emitiu alertas, como por exemplo sobre o \textit{.align} não poder estar dentro de um segmento de texto. Portanto todos \textit{.align} dentro de subrotinas foram retirados.

\begin{figure}[H]
	\centering
	\includegraphics[width=1\textwidth]{hilo.png}
	\caption{Instruções equivalentes dos construtores \%hi() e \%lo(). Imagem retirada do livro \cite{mipsrun}.}
	\label{fig:hilo}
\end{figure}

Os construtores \%hi() e \%lo() também não estão presentes em todos montadores MIPS, podendo ser substituídos como mostra a Figura~\ref{fig:hilo}. Logo, as seguintes instruções foram trocadas :

\begin{itemize}
	\item lui \$v1,\%hi(\$LC0) e addiu \$a0,\$v1,\%lo(\$LC0) : la \$a0, \$LC0
    \item lui \$v0,\%hi(v) e addiu \$a0,\$v0,\%lo(v) : la \$a0, v
\end{itemize}

Outro problema encontrado foi na instrução \textit{j \$31, ou seja, j \$ra}. A instrução \textit{j} pula para um endereço alvo, e \$ra é um registrador, portanto esta instrução foi substituída por \textit{jr \$ra}.

Algumas subrotinas como \textit{printf e putchar} não são encontradas no assembly gerado mas são chamadas com a instrução \textit{jal}, portanto estas também foram retiradas.

Por fim, mesmo após estas modificações básicas de sintaxe para que o código possa ser executado no MARS, ainda assim o código possui o fluxo um pouco confuso, já que após o primeiro \textit{loop} ele já iria sair do processo com a instrução \textit{li \$a0, 10} que chama o serviço \textit{exit (terminate execution)}. Portanto, algumas modificações foram realizadas para que o fluxo deste programa se inicie no \textit{main}, chame as subrotinas \textit{show} e \textit{sort} e encerre no final da subrotina \textit{main}.

\subsubsection{Otimização do código Assembly}
\label{subsubsec:ot}

O arquivo \textit{sortc.c} foi compilado novamente usando 5 diretivas de compilação, \textit{-O0, -O1, -O2, -O3 e -Os}. Todos os arquivos gerados, \textit{sortc0.s}, \textit{sortc1.s}, \textit{sortc2.s}, \textit{sortc3.s} e \textit{sortcs.s} foram alterados de acordo com a subseção anterior para que fosse possível executá-los no \textit{MARS}.

\subsection{Sprites}
\label{subsec:sprites}

	 

	As sprites utilizadas foram:
	
	\begin{figure}[H]
		\centering
		\begin{subfigure}{.5\textwidth}
			\centering
			\includegraphics[width=.125\textwidth]{parte1.jpg}
			\caption{Movimento 1 do Ryu.}
			\label{fig:Ryu1}
		\end{subfigure}
		\begin{subfigure}{.5\textwidth}
			\centering
			\includegraphics[width=.125\textwidth]{parte2.jpg}
			\caption{Movimento 2 do Ryu.}
			\label{fig:Ryu2}
		\end{subfigure}	
		\begin{subfigure}{.5\textwidth}
			\centering
			\includegraphics[width=.125\textwidth]{parte3.jpg}
			\caption{Movimento 3 do Ryu.}
			\label{fig:Ryu3}
		\end{subfigure}
		\begin{subfigure}{.5\textwidth}
			\centering
			\includegraphics[width=.125\textwidth]{parte4.jpg}
			\caption{Movimento 4 do Ryu.}
			\label{fig:Ryu4}
		\end{subfigure}	
		\label{fig:SpriteRyu}
	\end{figure}
	

\section{Análise dos Resultados}
\label{sec:Resultados}

\section{Conclusão}
\label{sec:Conclusao}

\bibliographystyle{sbc}
\bibliography{relatorio}

\end{document}
